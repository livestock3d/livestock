%% Generated by Sphinx.
\def\sphinxdocclass{report}
\documentclass[letterpaper,10pt,english]{sphinxmanual}
\ifdefined\pdfpxdimen
   \let\sphinxpxdimen\pdfpxdimen\else\newdimen\sphinxpxdimen
\fi \sphinxpxdimen=.75bp\relax

\usepackage[utf8]{inputenc}
\ifdefined\DeclareUnicodeCharacter
 \ifdefined\DeclareUnicodeCharacterAsOptional
  \DeclareUnicodeCharacter{"00A0}{\nobreakspace}
  \DeclareUnicodeCharacter{"2500}{\sphinxunichar{2500}}
  \DeclareUnicodeCharacter{"2502}{\sphinxunichar{2502}}
  \DeclareUnicodeCharacter{"2514}{\sphinxunichar{2514}}
  \DeclareUnicodeCharacter{"251C}{\sphinxunichar{251C}}
  \DeclareUnicodeCharacter{"2572}{\textbackslash}
 \else
  \DeclareUnicodeCharacter{00A0}{\nobreakspace}
  \DeclareUnicodeCharacter{2500}{\sphinxunichar{2500}}
  \DeclareUnicodeCharacter{2502}{\sphinxunichar{2502}}
  \DeclareUnicodeCharacter{2514}{\sphinxunichar{2514}}
  \DeclareUnicodeCharacter{251C}{\sphinxunichar{251C}}
  \DeclareUnicodeCharacter{2572}{\textbackslash}
 \fi
\fi
\usepackage{cmap}
\usepackage[T1]{fontenc}
\usepackage{amsmath,amssymb,amstext}
\usepackage{babel}
\usepackage{times}
\usepackage[Bjarne]{fncychap}
\usepackage[dontkeepoldnames]{sphinx}

\usepackage{geometry}

% Include hyperref last.
\usepackage{hyperref}
% Fix anchor placement for figures with captions.
\usepackage{hypcap}% it must be loaded after hyperref.
% Set up styles of URL: it should be placed after hyperref.
\urlstyle{same}

\addto\captionsenglish{\renewcommand{\figurename}{Fig.}}
\addto\captionsenglish{\renewcommand{\tablename}{Table}}
\addto\captionsenglish{\renewcommand{\literalblockname}{Listing}}

\addto\captionsenglish{\renewcommand{\literalblockcontinuedname}{continued from previous page}}
\addto\captionsenglish{\renewcommand{\literalblockcontinuesname}{continues on next page}}

\addto\extrasenglish{\def\pageautorefname{page}}

\setcounter{tocdepth}{1}



\title{Livestock CPython Package Documentation}
\date{Mar 11, 2018}
\release{2018.3}
\author{Christian Kongsgaard}
\newcommand{\sphinxlogo}{\vbox{}}
\renewcommand{\releasename}{Release}
\makeindex

\begin{document}

\maketitle
\sphinxtableofcontents
\phantomsection\label{\detokenize{index::doc}}


Livestock is the name of the library of components that has been developed for this thesis.
Livestock consists of a series of Grasshopper Python Script components and a underlying collection of Python scripts
and a PyPI \textendash{} Python Package Index - package. This is the documentation for the PyPI package.


\chapter{Documentation for the PyPI Package:}
\label{\detokenize{index:documentation-for-the-pypi-package}}\label{\detokenize{index:welcome-to-livestock-cpython-package-s-documentation}}

\section{Livestock Air}
\label{\detokenize{air:livestock-air}}\label{\detokenize{air::doc}}\label{\detokenize{air:module-livestock.air}}\index{livestock.air (module)}\index{celsius\_to\_kelvin() (in module livestock.air)}

\begin{fulllineitems}
\phantomsection\label{\detokenize{air:livestock.air.celsius_to_kelvin}}\pysiglinewithargsret{\sphinxcode{livestock.air.}\sphinxbfcode{celsius\_to\_kelvin}}{\emph{celsius: float}}{{ $\rightarrow$ float}}
Converts a temperature in Celsius to Kelvin.

Source: \sphinxurl{https://en.wikipedia.org/wiki/Celsius}
\begin{quote}\begin{description}
\item[{Parameters}] \leavevmode
\sphinxstyleliteralstrong{celsius} (\sphinxstyleliteralemphasis{float}) \textendash{} Temperature in Celsius

\item[{Returns}] \leavevmode
Temperature in Kelvin

\item[{Return type}] \leavevmode
float

\end{description}\end{quote}

\end{fulllineitems}

\index{compute\_temperature\_relative\_humidity() (in module livestock.air)}

\begin{fulllineitems}
\phantomsection\label{\detokenize{air:livestock.air.compute_temperature_relative_humidity}}\pysiglinewithargsret{\sphinxcode{livestock.air.}\sphinxbfcode{compute\_temperature\_relative\_humidity}}{\emph{temperature\_in\_k: \textless{}built-in function array\textgreater{}}, \emph{relative\_humidity: \textless{}built-in function array\textgreater{}}, \emph{vapour\_mass\_flux: \textless{}built-in function array\textgreater{}}, \emph{volume: \textless{}built-in function array\textgreater{}}}{{ $\rightarrow$ tuple}}
Computes the coupled relative humidity and temperature of an air volume given a vapour flux. The vapour pressure is
capped off so it can not exceed the saturated vapour pressure. This potential means that not the whole amount
\begin{quote}

of vapour flux will be used.
\end{quote}
\begin{quote}\begin{description}
\item[{Parameters}] \leavevmode\begin{itemize}
\item {} 
\sphinxstyleliteralstrong{temperature\_in\_k} (\sphinxstyleliteralemphasis{numpy.array}) \textendash{} Current temperature of the air volume in K

\item {} 
\sphinxstyleliteralstrong{relative\_humidity} (\sphinxstyleliteralemphasis{numpy.array}) \textendash{} Current relative humidity of the air volume as unit less.

\item {} 
\sphinxstyleliteralstrong{vapour\_mass\_flux} (\sphinxstyleliteralemphasis{numpy.array}) \textendash{} Vapour mass flux to be added to the air volume in kg/h.

\item {} 
\sphinxstyleliteralstrong{volume} \textendash{} Air volume in m$^{\text{3}}$    :type volume: numpy.array

\end{itemize}

\item[{Returns}] \leavevmode
Tuple containing the new temperature in K, new relative humidity as unit less, the latent heat used and
the vapour flux used.

\item[{Return type}] \leavevmode
tuple

\end{description}\end{quote}

\end{fulllineitems}

\index{convert\_relative\_humidity\_to\_percentage() (in module livestock.air)}

\begin{fulllineitems}
\phantomsection\label{\detokenize{air:livestock.air.convert_relative_humidity_to_percentage}}\pysiglinewithargsret{\sphinxcode{livestock.air.}\sphinxbfcode{convert\_relative\_humidity\_to\_percentage}}{\emph{rh: \textless{}built-in function array\textgreater{}}}{{ $\rightarrow$ \textless{}built-in function array\textgreater{}}}
Converts relative humidity from percentage to an unit less number.
\begin{quote}\begin{description}
\item[{Parameters}] \leavevmode
\sphinxstyleliteralstrong{rh} (\sphinxstyleliteralemphasis{numpy.array}) \textendash{} Relative humidity as unitless

\item[{Returns}] \leavevmode
Relative humidity in percentage

\item[{Return type}] \leavevmode
numpy.array

\end{description}\end{quote}

\end{fulllineitems}

\index{convert\_relative\_humidity\_to\_unitless() (in module livestock.air)}

\begin{fulllineitems}
\phantomsection\label{\detokenize{air:livestock.air.convert_relative_humidity_to_unitless}}\pysiglinewithargsret{\sphinxcode{livestock.air.}\sphinxbfcode{convert\_relative\_humidity\_to\_unitless}}{\emph{rh: \textless{}built-in function array\textgreater{}}}{{ $\rightarrow$ \textless{}built-in function array\textgreater{}}}
Converts relative humidity from percentage to an unit less number.
\begin{quote}\begin{description}
\item[{Parameters}] \leavevmode
\sphinxstyleliteralstrong{rh} (\sphinxstyleliteralemphasis{numpy.array}) \textendash{} Relative humidity in \%

\item[{Returns}] \leavevmode
Relative humidity as unitless

\item[{Return type}] \leavevmode
numpy.array

\end{description}\end{quote}

\end{fulllineitems}

\index{convert\_vapour\_flux\_to\_kgh() (in module livestock.air)}

\begin{fulllineitems}
\phantomsection\label{\detokenize{air:livestock.air.convert_vapour_flux_to_kgh}}\pysiglinewithargsret{\sphinxcode{livestock.air.}\sphinxbfcode{convert\_vapour\_flux\_to\_kgh}}{\emph{vapour\_flux: \textless{}built-in function array\textgreater{}}}{{ $\rightarrow$ \textless{}built-in function array\textgreater{}}}
Converts a vapour flux from m$^{\text{3}}$/day to kg/h
Density of water: 1000kg/m$^{\text{3}}$    Hours per day: 24h/day
Conversion: 1000kg/m$^{\text{3}}$/ 24h/day
\begin{quote}\begin{description}
\item[{Parameters}] \leavevmode
\sphinxstyleliteralstrong{vapour\_flux} (\sphinxstyleliteralemphasis{numpy.array}) \textendash{} Vapour flux in m$^{\text{3}}$/day

\item[{Returns}] \leavevmode
Vapour flux in kg/h

\item[{Return type}] \leavevmode
numpy.array

\end{description}\end{quote}

\end{fulllineitems}

\index{diameter\_from\_area() (in module livestock.air)}

\begin{fulllineitems}
\phantomsection\label{\detokenize{air:livestock.air.diameter_from_area}}\pysiglinewithargsret{\sphinxcode{livestock.air.}\sphinxbfcode{diameter\_from\_area}}{\emph{area: \textless{}built-in function array\textgreater{}}}{{ $\rightarrow$ \textless{}built-in function array\textgreater{}}}
Computes the diameter from a given area of a circle.
A = \({\pi}\) * (d/2)$^{\text{2}}$ =\textgreater{}
d = \({\sqrt{4*A/{\pi}}}\)
\begin{quote}\begin{description}
\item[{Parameters}] \leavevmode
\sphinxstyleliteralstrong{area} (\sphinxstyleliteralemphasis{numpy.array}) \textendash{} Area of a circle in m

\item[{Returns}] \leavevmode
Diameter in m

\item[{Return type}] \leavevmode
numpy.array

\end{description}\end{quote}

\end{fulllineitems}

\index{kelvin\_to\_celsius() (in module livestock.air)}

\begin{fulllineitems}
\phantomsection\label{\detokenize{air:livestock.air.kelvin_to_celsius}}\pysiglinewithargsret{\sphinxcode{livestock.air.}\sphinxbfcode{kelvin\_to\_celsius}}{\emph{kelvin: float}}{{ $\rightarrow$ float}}
Converts a temperature in Kelvin to Celsius.

Source: \sphinxurl{https://en.wikipedia.org/wiki/Celsius}
\begin{quote}\begin{description}
\item[{Parameters}] \leavevmode
\sphinxstyleliteralstrong{kelvin} (\sphinxstyleliteralemphasis{float}) \textendash{} Temperature in Kelvin

\item[{Returns}] \leavevmode
Temperature in Celsius

\item[{Return type}] \leavevmode
float

\end{description}\end{quote}

\end{fulllineitems}

\index{latent\_heat\_flux() (in module livestock.air)}

\begin{fulllineitems}
\phantomsection\label{\detokenize{air:livestock.air.latent_heat_flux}}\pysiglinewithargsret{\sphinxcode{livestock.air.}\sphinxbfcode{latent\_heat\_flux}}{\emph{vapour\_mass\_flux: \textless{}built-in function array\textgreater{}}}{{ $\rightarrow$ \textless{}built-in function array\textgreater{}}}
Computes the latent heat flux related to a certain evapotranspiration flux.
The latent heat flux is negative if the vapour flux is positive.

Source: Manickathan, L. et al., 2018.
Parametric study of the influence of environmental factors and tree properties on the
transpirative cooling effect of trees. Agricultural and Forest Meteorology.
\begin{quote}\begin{description}
\item[{Parameters}] \leavevmode
\sphinxstyleliteralstrong{vapour\_mass\_flux} (\sphinxstyleliteralemphasis{numpy.array}) \textendash{} Vapour volume flux in kg/h

\item[{Returns}] \leavevmode
Latent heat flux in J/h.

\item[{Return type}] \leavevmode
numpy.array

\end{description}\end{quote}

\end{fulllineitems}

\index{max\_possible\_vapour\_flux() (in module livestock.air)}

\begin{fulllineitems}
\phantomsection\label{\detokenize{air:livestock.air.max_possible_vapour_flux}}\pysiglinewithargsret{\sphinxcode{livestock.air.}\sphinxbfcode{max\_possible\_vapour\_flux}}{\emph{vapour\_mass\_flux: float}, \emph{volume: float}, \emph{temperature\_in\_kelvin: float}, \emph{vapour\_pressure: float}}{{ $\rightarrow$ float}}
Computes the difference between the saturated vapour pressure of an air volume after adding the vapour
and latent heat flux to an air volume and the actual vapour pressure of an air volume.
\begin{quote}\begin{description}
\item[{Parameters}] \leavevmode\begin{itemize}
\item {} 
\sphinxstyleliteralstrong{vapour\_mass\_flux} (\sphinxstyleliteralemphasis{float}) \textendash{} Vapour mass flux in kg/h

\item {} 
\sphinxstyleliteralstrong{volume} \textendash{} Air volume in m$^{\text{3}}$    :type volume: float

\item {} 
\sphinxstyleliteralstrong{temperature\_in\_kelvin} (\sphinxstyleliteralemphasis{float}) \textendash{} Current temperature in K

\item {} 
\sphinxstyleliteralstrong{vapour\_pressure} (\sphinxstyleliteralemphasis{float}) \textendash{} Current vapour pressure in Pa

\end{itemize}

\item[{Returns}] \leavevmode
Difference between the saturated vapour pressure after adding the vapour and latent heat flux to

\end{description}\end{quote}

the air volume and the actual vapour pressure of the air volume.
:rtype: float

\end{fulllineitems}

\index{new\_mean\_relative\_humidity() (in module livestock.air)}

\begin{fulllineitems}
\phantomsection\label{\detokenize{air:livestock.air.new_mean_relative_humidity}}\pysiglinewithargsret{\sphinxcode{livestock.air.}\sphinxbfcode{new\_mean\_relative\_humidity}}{\emph{volume: \textless{}built-in function array\textgreater{}}, \emph{temperature\_internal: \textless{}built-in function array\textgreater{}}, \emph{vapour\_pressure\_external: \textless{}built-in function array\textgreater{}}, \emph{vapour\_production: \textless{}built-in function array\textgreater{}}}{{ $\rightarrow$ \textless{}built-in function array\textgreater{}}}
Computes a new mean vapour pressure and converts it in to a relative humidity.

Source: Peuhkuri, Ruut, and Carsten Rode. 2016.
“Heat and Mass Transfer in Buildings.”
\begin{quote}\begin{description}
\item[{Parameters}] \leavevmode\begin{itemize}
\item {} 
\sphinxstyleliteralstrong{volume} (\sphinxstyleliteralemphasis{numpy.array}) \textendash{} Air volume in m$^{\text{3}}$

\item {} 
\sphinxstyleliteralstrong{temperature\_internal} (\sphinxstyleliteralemphasis{numpy.array}) \textendash{} External temperature in K

\item {} 
\sphinxstyleliteralstrong{vapour\_pressure\_external} (\sphinxstyleliteralemphasis{numpy.array}) \textendash{} External vapour pressure in Pa

\item {} 
\sphinxstyleliteralstrong{vapour\_production} (\sphinxstyleliteralemphasis{numpy.array}) \textendash{} Vapour production in kg/h

\end{itemize}

\item[{Returns}] \leavevmode
Relative humidity - unitless

\item[{Return type}] \leavevmode
numpy.array

\end{description}\end{quote}

\end{fulllineitems}

\index{new\_mean\_temperature() (in module livestock.air)}

\begin{fulllineitems}
\phantomsection\label{\detokenize{air:livestock.air.new_mean_temperature}}\pysiglinewithargsret{\sphinxcode{livestock.air.}\sphinxbfcode{new\_mean\_temperature}}{\emph{volume: \textless{}built-in function array\textgreater{}}, \emph{temperature: \textless{}built-in function array\textgreater{}}, \emph{heat: \textless{}built-in function array\textgreater{}}}{{ $\rightarrow$ \textless{}built-in function array\textgreater{}}}
Calculates a new mean temperature for the volume.

Source: Peuhkuri, Ruut, and Carsten Rode. 2016.
“Heat and Mass Transfer in Buildings.”
\begin{quote}\begin{description}
\item[{Parameters}] \leavevmode\begin{itemize}
\item {} 
\sphinxstyleliteralstrong{volume} (\sphinxstyleliteralemphasis{numpy.array}) \textendash{} Volume in m$^{\text{3}}$

\item {} 
\sphinxstyleliteralstrong{temperature} (\sphinxstyleliteralemphasis{numpy.array}) \textendash{} Temperature at the top of the air volume in K

\item {} 
\sphinxstyleliteralstrong{heat} (\sphinxstyleliteralemphasis{numpy.array}) \textendash{} Added heat to the air volume in J/h

\end{itemize}

\item[{Returns}] \leavevmode
Temperature in K

\item[{Return type}] \leavevmode
numpy.array

\end{description}\end{quote}

\end{fulllineitems}

\index{new\_mean\_vapour\_pressure() (in module livestock.air)}

\begin{fulllineitems}
\phantomsection\label{\detokenize{air:livestock.air.new_mean_vapour_pressure}}\pysiglinewithargsret{\sphinxcode{livestock.air.}\sphinxbfcode{new\_mean\_vapour\_pressure}}{\emph{volume: \textless{}built-in function array\textgreater{}}, \emph{temperature: \textless{}built-in function array\textgreater{}}, \emph{vapour\_pressure\_external: \textless{}built-in function array\textgreater{}}, \emph{vapour\_production: \textless{}built-in function array\textgreater{}}}{{ $\rightarrow$ \textless{}built-in function array\textgreater{}}}
Calculates a new vapour pressure for the volume.

Source: Peuhkuri, Ruut, and Carsten Rode. 2016.
“Heat and Mass Transfer in Buildings.”
\begin{quote}\begin{description}
\item[{Parameters}] \leavevmode\begin{itemize}
\item {} 
\sphinxstyleliteralstrong{volume} (\sphinxstyleliteralemphasis{numpy.array}) \textendash{} Volume in m$^{\text{3}}$

\item {} 
\sphinxstyleliteralstrong{temperature} (\sphinxstyleliteralemphasis{numpy.array}) \textendash{} Temperature in K

\item {} 
\sphinxstyleliteralstrong{vapour\_pressure\_external} (\sphinxstyleliteralemphasis{numpy.array}) \textendash{} External vapour pressure in Pa

\item {} 
\sphinxstyleliteralstrong{vapour\_production} (\sphinxstyleliteralemphasis{numpy.array}) \textendash{} Vapour production in kg/h

\end{itemize}

\item[{Returns}] \leavevmode
New vapour pressure in Pa

\item[{Return type}] \leavevmode
numpy.array

\end{description}\end{quote}

\end{fulllineitems}

\index{new\_temperature\_and\_relative\_humidity() (in module livestock.air)}

\begin{fulllineitems}
\phantomsection\label{\detokenize{air:livestock.air.new_temperature_and_relative_humidity}}\pysiglinewithargsret{\sphinxcode{livestock.air.}\sphinxbfcode{new\_temperature\_and\_relative\_humidity}}{\emph{folder: str}}{{ $\rightarrow$ bool}}
Calculates a new temperatures and relative humidity for air volumes.
\begin{quote}\begin{description}
\item[{Parameters}] \leavevmode
\sphinxstyleliteralstrong{folder} (\sphinxstyleliteralemphasis{str}) \textendash{} Path to folder containing case files.

\item[{Returns}] \leavevmode
True

\item[{Return type}] \leavevmode
bool

\end{description}\end{quote}

\end{fulllineitems}

\index{relative\_humidity\_to\_vapour\_pressure() (in module livestock.air)}

\begin{fulllineitems}
\phantomsection\label{\detokenize{air:livestock.air.relative_humidity_to_vapour_pressure}}\pysiglinewithargsret{\sphinxcode{livestock.air.}\sphinxbfcode{relative\_humidity\_to\_vapour\_pressure}}{\emph{relative\_humidity: float}, \emph{temperature: float}}{{ $\rightarrow$ float}}
Convert relative humidity to vapour pressure given a air temperature.

Source: Peuhkuri, Ruut, and Carsten Rode. 2016.
“Heat and Mass Transfer in Buildings.”
\begin{quote}\begin{description}
\item[{Parameters}] \leavevmode\begin{itemize}
\item {} 
\sphinxstyleliteralstrong{relative\_humidity} (\sphinxstyleliteralemphasis{float}) \textendash{} Relative humidity - unitless

\item {} 
\sphinxstyleliteralstrong{temperature} (\sphinxstyleliteralemphasis{float}) \textendash{} Air temperature in K

\end{itemize}

\item[{Returns}] \leavevmode
Vapour pressure in Pa

\item[{Return type}] \leavevmode
float

\end{description}\end{quote}

\end{fulllineitems}

\index{run\_row() (in module livestock.air)}

\begin{fulllineitems}
\phantomsection\label{\detokenize{air:livestock.air.run_row}}\pysiglinewithargsret{\sphinxcode{livestock.air.}\sphinxbfcode{run\_row}}{\emph{input\_package: list}}{{ $\rightarrow$ tuple}}
Calculates a new temperatures and relative humidity for a row. A row represent all cells to a given time.
\begin{quote}\begin{description}
\item[{Parameters}] \leavevmode
\sphinxstyleliteralstrong{input\_package} (\sphinxstyleliteralemphasis{list}) \textendash{} Input package with need inputs.

\item[{Returns}] \leavevmode
The row on which the calculation was performed.

\item[{Return type}] \leavevmode
tuple

\end{description}\end{quote}

\end{fulllineitems}

\index{saturated\_vapour\_pressure() (in module livestock.air)}

\begin{fulllineitems}
\phantomsection\label{\detokenize{air:livestock.air.saturated_vapour_pressure}}\pysiglinewithargsret{\sphinxcode{livestock.air.}\sphinxbfcode{saturated\_vapour\_pressure}}{\emph{temperature: float}}{{ $\rightarrow$ float}}
Computes the saturated vapour pressure for a given temperature.
Source: Peuhkuri, Ruut, and Carsten Rode. 2016.
“Heat and Mass Transfer in Buildings.”
\begin{quote}\begin{description}
\item[{Parameters}] \leavevmode
\sphinxstyleliteralstrong{temperature} (\sphinxstyleliteralemphasis{float}) \textendash{} Temperature in Kelvin

\item[{Returns}] \leavevmode
Vapour pressure in Pa

\item[{Return type}] \leavevmode
float

\end{description}\end{quote}

\end{fulllineitems}

\index{stratification() (in module livestock.air)}

\begin{fulllineitems}
\phantomsection\label{\detokenize{air:livestock.air.stratification}}\pysiglinewithargsret{\sphinxcode{livestock.air.}\sphinxbfcode{stratification}}{\emph{height: float}, \emph{value\_mean: float}, \emph{height\_top: float}, \emph{value\_top: float}}{{ $\rightarrow$ float}}
Calculates the stratification of the temperature or relative humidity of the air volume.
\begin{quote}\begin{description}
\item[{Parameters}] \leavevmode\begin{itemize}
\item {} 
\sphinxstyleliteralstrong{height} (\sphinxstyleliteralemphasis{float}) \textendash{} Height at which the stratification value is wanted in m.

\item {} 
\sphinxstyleliteralstrong{value\_mean} (\sphinxstyleliteralemphasis{float}) \textendash{} Mean value of the air volume. Assumed equal to the value at half of the height of the air volume.

\item {} 
\sphinxstyleliteralstrong{height\_top} (\sphinxstyleliteralemphasis{float}) \textendash{} Height at the top of the boundary in m.

\item {} 
\sphinxstyleliteralstrong{value\_top} (\sphinxstyleliteralemphasis{float}) \textendash{} Value at the top of the air volume

\end{itemize}

\item[{Returns}] \leavevmode
Value at desired height.

\item[{Return type}] \leavevmode
float

\end{description}\end{quote}

\end{fulllineitems}

\index{vapour\_pressure\_to\_relative\_humidity() (in module livestock.air)}

\begin{fulllineitems}
\phantomsection\label{\detokenize{air:livestock.air.vapour_pressure_to_relative_humidity}}\pysiglinewithargsret{\sphinxcode{livestock.air.}\sphinxbfcode{vapour\_pressure\_to\_relative\_humidity}}{\emph{vapour\_pressure: float}, \emph{temperature: float}}{{ $\rightarrow$ float}}
Convert vapour pressure to relative humidity given a air temperature

Source: Peuhkuri, Ruut, and Carsten Rode. 2016.
“Heat and Mass Transfer in Buildings.”
\begin{quote}\begin{description}
\item[{Parameters}] \leavevmode\begin{itemize}
\item {} 
\sphinxstyleliteralstrong{vapour\_pressure} (\sphinxstyleliteralemphasis{float}) \textendash{} Vapour pressure in Pa

\item {} 
\sphinxstyleliteralstrong{temperature} (\sphinxstyleliteralemphasis{float}) \textendash{} Air temperature in K

\end{itemize}

\item[{Returns}] \leavevmode
Relative humidity as unitless

\item[{Return type}] \leavevmode
float

\end{description}\end{quote}

\end{fulllineitems}

\index{wind\_speed\_to\_flux() (in module livestock.air)}

\begin{fulllineitems}
\phantomsection\label{\detokenize{air:livestock.air.wind_speed_to_flux}}\pysiglinewithargsret{\sphinxcode{livestock.air.}\sphinxbfcode{wind\_speed\_to\_flux}}{\emph{wind\_speed: \textless{}built-in function array\textgreater{}}, \emph{height: \textless{}built-in function array\textgreater{}}, \emph{cross\_section: \textless{}built-in function array\textgreater{}}}{{ $\rightarrow$ \textless{}built-in function array\textgreater{}}}
Converts a wind speed through an area to a wind flux.
\begin{quote}\begin{description}
\item[{Parameters}] \leavevmode\begin{itemize}
\item {} 
\sphinxstyleliteralstrong{wind\_speed} (\sphinxstyleliteralemphasis{numpy.array}) \textendash{} Wind speed in m/s

\item {} 
\sphinxstyleliteralstrong{height} (\sphinxstyleliteralemphasis{numpy.array}) \textendash{} Height of the area in m

\item {} 
\sphinxstyleliteralstrong{cross\_section} (\sphinxstyleliteralemphasis{numpy.array}) \textendash{} Width of the area in m

\end{itemize}

\item[{Returns}] \leavevmode
Wind flux in m$^{\text{3}}$/h

\item[{Return type}] \leavevmode
numpy.array

\end{description}\end{quote}

\end{fulllineitems}

\index{wind\_speed\_to\_hour\_flux() (in module livestock.air)}

\begin{fulllineitems}
\phantomsection\label{\detokenize{air:livestock.air.wind_speed_to_hour_flux}}\pysiglinewithargsret{\sphinxcode{livestock.air.}\sphinxbfcode{wind\_speed\_to\_hour\_flux}}{\emph{wind\_speed: float}}{{ $\rightarrow$ float}}
Converts wind speed into a hourly flux.
m/s to m$^{\text{3}}$/h
m/s to m$^{\text{3}}$/s = 1:sup:\sphinxtitleref{2}
m$^{\text{3}}$/s to m$^{\text{3}}$/h = 3600s/h
\begin{quote}\begin{description}
\item[{Parameters}] \leavevmode
\sphinxstyleliteralstrong{wind\_speed} (\sphinxstyleliteralemphasis{float}) \textendash{} Wind speed in m/s

\item[{Returns}] \leavevmode
Wind flux in m$^{\text{3}}$/h

\item[{Return type}] \leavevmode
float

\end{description}\end{quote}

\end{fulllineitems}


\sphinxstylestrong{Go Back to:}

\sphinxhref{https://ocni-dtu.github.io/}{Livestock Frontpage}

\sphinxhref{https://ocni-dtu.github.io/livestock/index.html}{Livestock PyPi}

\sphinxhref{https://ocni-dtu.github.io/livestock\_gh/index.html}{Livestock Grasshopper}


\section{Livestock Geometry}
\label{\detokenize{geometry:livestock-geometry}}\label{\detokenize{geometry::doc}}\label{\detokenize{geometry:id3}}\label{\detokenize{geometry:module-livestock.geometry}}\index{livestock.geometry (module)}\index{centroid\_z() (in module livestock.geometry)}

\begin{fulllineitems}
\phantomsection\label{\detokenize{geometry:livestock.geometry.centroid_z}}\pysiglinewithargsret{\sphinxcode{livestock.geometry.}\sphinxbfcode{centroid\_z}}{\emph{polygon: shapely.geometry.polygon.Polygon}}{{ $\rightarrow$ float}}
Calculates the mean z-value from a Shapely polygon.
\begin{quote}\begin{description}
\item[{Parameters}] \leavevmode
\sphinxstyleliteralstrong{polygon} (\sphinxstyleliteralemphasis{shapely.geometry.Polygon}) \textendash{} Shapely Polygon with z-values.

\item[{Returns}] \leavevmode
Mean z-value of the polygon

\item[{Return type}] \leavevmode
float

\end{description}\end{quote}

\end{fulllineitems}

\index{obj\_to\_lists() (in module livestock.geometry)}

\begin{fulllineitems}
\phantomsection\label{\detokenize{geometry:livestock.geometry.obj_to_lists}}\pysiglinewithargsret{\sphinxcode{livestock.geometry.}\sphinxbfcode{obj\_to\_lists}}{\emph{obj\_file: str}}{{ $\rightarrow$ tuple}}
Converts an .obj file into lists.
\begin{quote}\begin{description}
\item[{Parameters}] \leavevmode
\sphinxstyleliteralstrong{obj\_file} (\sphinxstyleliteralemphasis{str}) \textendash{} .obj file path

\item[{Returns}] \leavevmode
tuple with vertices, normals, faces

\item[{Return type}] \leavevmode
tuple

\end{description}\end{quote}

\end{fulllineitems}

\index{obj\_to\_polygons() (in module livestock.geometry)}

\begin{fulllineitems}
\phantomsection\label{\detokenize{geometry:livestock.geometry.obj_to_polygons}}\pysiglinewithargsret{\sphinxcode{livestock.geometry.}\sphinxbfcode{obj\_to\_polygons}}{\emph{obj\_file: str}}{{ $\rightarrow$ list}}
Converts an .obj file into a list of shapely polygons.
\begin{quote}\begin{description}
\item[{Parameters}] \leavevmode
\sphinxstyleliteralstrong{obj\_file} (\sphinxstyleliteralemphasis{str}) \textendash{} .obj file path

\item[{Returns}] \leavevmode
Shapely polygons in a list

\item[{Return type}] \leavevmode
list

\end{description}\end{quote}

\end{fulllineitems}

\index{obj\_to\_shp() (in module livestock.geometry)}

\begin{fulllineitems}
\phantomsection\label{\detokenize{geometry:livestock.geometry.obj_to_shp}}\pysiglinewithargsret{\sphinxcode{livestock.geometry.}\sphinxbfcode{obj\_to\_shp}}{\emph{obj\_file: str}, \emph{shp\_file: str}}{{ $\rightarrow$ bool}}
Convert an .obj file into a shape file.
\begin{quote}\begin{description}
\item[{Parameters}] \leavevmode\begin{itemize}
\item {} 
\sphinxstyleliteralstrong{obj\_file} (\sphinxstyleliteralemphasis{str}) \textendash{} Path to .obj file

\item {} 
\sphinxstyleliteralstrong{shp\_file} (\sphinxstyleliteralemphasis{str}) \textendash{} File path for shapefile

\end{itemize}

\item[{Returns}] \leavevmode
True

\item[{Return type}] \leavevmode
bool

\end{description}\end{quote}

\end{fulllineitems}

\index{shapely\_to\_pyshp() (in module livestock.geometry)}

\begin{fulllineitems}
\phantomsection\label{\detokenize{geometry:livestock.geometry.shapely_to_pyshp}}\pysiglinewithargsret{\sphinxcode{livestock.geometry.}\sphinxbfcode{shapely\_to\_pyshp}}{\emph{shapely\_geometry: shapely.geometry.polygon.Polygon}}{{ $\rightarrow$ shapefile.\_Shape}}~
\begin{DUlineblock}{0em}
\item[] This function converts a shapely geometry into a geojson and then into a pyshp object.
\item[] Copied from Karim Bahgat’s answer at:
\item[] \sphinxurl{https://gis.stackexchange.com/questions/52705/how-to-write-shapely-geometries-to-shapefiles}
\item[] 
\end{DUlineblock}
\begin{quote}\begin{description}
\item[{Parameters}] \leavevmode
\sphinxstyleliteralstrong{shapely\_geometry} (\sphinxstyleliteralemphasis{shapely.geometry}) \textendash{} Shapely geometry to convert.

\item[{Returns}] \leavevmode
pyshp record object

\item[{Return type}] \leavevmode
shapefile.\_Shape

\end{description}\end{quote}

\end{fulllineitems}


\sphinxstylestrong{Go Back to:}

\sphinxhref{https://ocni-dtu.github.io/}{Livestock Frontpage}

\sphinxhref{https://ocni-dtu.github.io/livestock/index.html}{Livestock PyPi}

\sphinxhref{https://ocni-dtu.github.io/livestock\_gh/index.html}{Livestock Grasshopper}


\section{Livestock Hydrology}
\label{\detokenize{hydrology:module-livestock.hydrology}}\label{\detokenize{hydrology::doc}}\label{\detokenize{hydrology:id3}}\label{\detokenize{hydrology:livestock-hydrology}}\index{livestock.hydrology (module)}\index{CMFModel (class in livestock.hydrology)}

\begin{fulllineitems}
\phantomsection\label{\detokenize{hydrology:livestock.hydrology.CMFModel}}\pysiglinewithargsret{\sphinxbfcode{class }\sphinxcode{livestock.hydrology.}\sphinxbfcode{CMFModel}}{\emph{folder}}{}
Bases: \sphinxcode{object}
\index{add\_tree() (livestock.hydrology.CMFModel method)}

\begin{fulllineitems}
\phantomsection\label{\detokenize{hydrology:livestock.hydrology.CMFModel.add_tree}}\pysiglinewithargsret{\sphinxbfcode{add\_tree}}{\emph{cmf\_project}, \emph{cell\_index}, \emph{property\_dict}}{}
Adds a tree to the model

\end{fulllineitems}

\index{config\_outputs() (livestock.hydrology.CMFModel method)}

\begin{fulllineitems}
\phantomsection\label{\detokenize{hydrology:livestock.hydrology.CMFModel.config_outputs}}\pysiglinewithargsret{\sphinxbfcode{config\_outputs}}{\emph{cmf\_project}}{}
Function to set up result gathering dictionary

\end{fulllineitems}

\index{configure\_cells() (livestock.hydrology.CMFModel method)}

\begin{fulllineitems}
\phantomsection\label{\detokenize{hydrology:livestock.hydrology.CMFModel.configure_cells}}\pysiglinewithargsret{\sphinxbfcode{configure\_cells}}{\emph{cmf\_project: cmf.cmf\_core.project}, \emph{cell\_properties\_dict: dict}}{}
Configure the cells

\end{fulllineitems}

\index{create\_boundary\_conditions() (livestock.hydrology.CMFModel method)}

\begin{fulllineitems}
\phantomsection\label{\detokenize{hydrology:livestock.hydrology.CMFModel.create_boundary_conditions}}\pysiglinewithargsret{\sphinxbfcode{create\_boundary\_conditions}}{\emph{cmf\_project}}{}
\end{fulllineitems}

\index{create\_stream() (livestock.hydrology.CMFModel method)}

\begin{fulllineitems}
\phantomsection\label{\detokenize{hydrology:livestock.hydrology.CMFModel.create_stream}}\pysiglinewithargsret{\sphinxbfcode{create\_stream}}{\emph{shape}, \emph{shape\_param}, \emph{outlet}}{}
Create a stream

\end{fulllineitems}

\index{create\_weather() (livestock.hydrology.CMFModel method)}

\begin{fulllineitems}
\phantomsection\label{\detokenize{hydrology:livestock.hydrology.CMFModel.create_weather}}\pysiglinewithargsret{\sphinxbfcode{create\_weather}}{\emph{cmf\_project}}{}
Creates weather for the project

\end{fulllineitems}

\index{gather\_results() (livestock.hydrology.CMFModel method)}

\begin{fulllineitems}
\phantomsection\label{\detokenize{hydrology:livestock.hydrology.CMFModel.gather_results}}\pysiglinewithargsret{\sphinxbfcode{gather\_results}}{\emph{cmf\_project}, \emph{time}}{}
\end{fulllineitems}

\index{load\_cmf\_files() (livestock.hydrology.CMFModel method)}

\begin{fulllineitems}
\phantomsection\label{\detokenize{hydrology:livestock.hydrology.CMFModel.load_cmf_files}}\pysiglinewithargsret{\sphinxbfcode{load\_cmf\_files}}{\emph{delete\_after\_load=False}}{}
\end{fulllineitems}

\index{mesh\_to\_cells() (livestock.hydrology.CMFModel method)}

\begin{fulllineitems}
\phantomsection\label{\detokenize{hydrology:livestock.hydrology.CMFModel.mesh_to_cells}}\pysiglinewithargsret{\sphinxbfcode{mesh\_to\_cells}}{\emph{cmf\_project}, \emph{mesh\_path}, \emph{delete\_after\_load=True}}{}
Takes a mesh and converts it into CMF cells
:param mesh\_path: Path to mesh .obj file
:param cmf\_project: CMF project object.
:param delete\_after\_load: If True, it deletes the input files after they have been loaded.
:return: True

\end{fulllineitems}

\index{print\_solver\_time() (livestock.hydrology.CMFModel method)}

\begin{fulllineitems}
\phantomsection\label{\detokenize{hydrology:livestock.hydrology.CMFModel.print_solver_time}}\pysiglinewithargsret{\sphinxbfcode{print\_solver\_time}}{\emph{solver\_time}, \emph{start\_time}, \emph{last\_time}, \emph{step}}{}
\end{fulllineitems}

\index{run\_model() (livestock.hydrology.CMFModel method)}

\begin{fulllineitems}
\phantomsection\label{\detokenize{hydrology:livestock.hydrology.CMFModel.run_model}}\pysiglinewithargsret{\sphinxbfcode{run\_model}}{}{}
Runs the model with everything

\end{fulllineitems}

\index{save\_results() (livestock.hydrology.CMFModel method)}

\begin{fulllineitems}
\phantomsection\label{\detokenize{hydrology:livestock.hydrology.CMFModel.save_results}}\pysiglinewithargsret{\sphinxbfcode{save\_results}}{}{}
Saves the computed results to a xml file

\end{fulllineitems}

\index{set\_vegetation\_properties() (livestock.hydrology.CMFModel method)}

\begin{fulllineitems}
\phantomsection\label{\detokenize{hydrology:livestock.hydrology.CMFModel.set_vegetation_properties}}\pysiglinewithargsret{\sphinxbfcode{set\_vegetation\_properties}}{\emph{cell\_: cmf.cmf\_core.Cell}, \emph{property\_dict: dict}}{}
\end{fulllineitems}

\index{solve() (livestock.hydrology.CMFModel method)}

\begin{fulllineitems}
\phantomsection\label{\detokenize{hydrology:livestock.hydrology.CMFModel.solve}}\pysiglinewithargsret{\sphinxbfcode{solve}}{\emph{cmf\_project}, \emph{tolerance}}{}
Solves the model

\end{fulllineitems}


\end{fulllineitems}

\index{cell\_results() (in module livestock.hydrology)}

\begin{fulllineitems}
\phantomsection\label{\detokenize{hydrology:livestock.hydrology.cell_results}}\pysiglinewithargsret{\sphinxcode{livestock.hydrology.}\sphinxbfcode{cell\_results}}{\emph{looking\_for}, \emph{result\_file}, \emph{folder}}{}
Processes cell results

\end{fulllineitems}

\index{cmf\_results() (in module livestock.hydrology)}

\begin{fulllineitems}
\phantomsection\label{\detokenize{hydrology:livestock.hydrology.cmf_results}}\pysiglinewithargsret{\sphinxcode{livestock.hydrology.}\sphinxbfcode{cmf\_results}}{\emph{path}}{}
\end{fulllineitems}

\index{convert\_cmf\_points() (in module livestock.hydrology)}

\begin{fulllineitems}
\phantomsection\label{\detokenize{hydrology:livestock.hydrology.convert_cmf_points}}\pysiglinewithargsret{\sphinxcode{livestock.hydrology.}\sphinxbfcode{convert\_cmf\_points}}{\emph{points}}{}
\end{fulllineitems}

\index{layer\_results() (in module livestock.hydrology)}

\begin{fulllineitems}
\phantomsection\label{\detokenize{hydrology:livestock.hydrology.layer_results}}\pysiglinewithargsret{\sphinxcode{livestock.hydrology.}\sphinxbfcode{layer\_results}}{\emph{looking\_for}, \emph{result\_file}, \emph{folder}}{}
Processes layer results

\end{fulllineitems}

\index{surface\_flux\_results() (in module livestock.hydrology)}

\begin{fulllineitems}
\phantomsection\label{\detokenize{hydrology:livestock.hydrology.surface_flux_results}}\pysiglinewithargsret{\sphinxcode{livestock.hydrology.}\sphinxbfcode{surface\_flux\_results}}{\emph{path}}{}
\end{fulllineitems}


\sphinxstylestrong{Go Back to:}

\sphinxhref{https://ocni-dtu.github.io/}{Livestock Frontpage}

\sphinxhref{https://ocni-dtu.github.io/livestock/index.html}{Livestock PyPi}

\sphinxhref{https://ocni-dtu.github.io/livestock\_gh/index.html}{Livestock Grasshopper}


\section{Livestock Misc}
\label{\detokenize{misc:livestock-misc}}\label{\detokenize{misc::doc}}\label{\detokenize{misc:id3}}\label{\detokenize{misc:module-livestock.misc}}\index{livestock.misc (module)}\index{run\_cfd() (in module livestock.misc)}

\begin{fulllineitems}
\phantomsection\label{\detokenize{misc:livestock.misc.run_cfd}}\pysiglinewithargsret{\sphinxcode{livestock.misc.}\sphinxbfcode{run\_cfd}}{\emph{files\_path}}{}
Runs a OpenFoam case

\end{fulllineitems}


\sphinxstylestrong{Go Back to:}

\sphinxhref{https://ocni-dtu.github.io/}{Livestock Frontpage}

\sphinxhref{https://ocni-dtu.github.io/livestock/index.html}{Livestock PyPi}

\sphinxhref{https://ocni-dtu.github.io/livestock\_gh/index.html}{Livestock Grasshopper}


\section{Livestock SSH}
\label{\detokenize{ssh:module-livestock.ssh}}\label{\detokenize{ssh::doc}}\label{\detokenize{ssh:id3}}\label{\detokenize{ssh:livestock-ssh}}\index{livestock.ssh (module)}\index{check\_for\_remote\_folder() (in module livestock.ssh)}

\begin{fulllineitems}
\phantomsection\label{\detokenize{ssh:livestock.ssh.check_for_remote_folder}}\pysiglinewithargsret{\sphinxcode{livestock.ssh.}\sphinxbfcode{check\_for\_remote\_folder}}{\emph{sftp\_connect: \textless{}function SSHClient.open\_sftp at 0x0000025AD7132620\textgreater{}}, \emph{folder\_to\_check: str}, \emph{check\_for: str}}{{ $\rightarrow$ bool}}
Checks if remote folder exists in the desired location. If do exists the function returns True.
Otherwise is creates the folder and then returns True.
\begin{quote}\begin{description}
\item[{Parameters}] \leavevmode\begin{itemize}
\item {} 
\sphinxstyleliteralstrong{sftp\_connect} (\sphinxstyleliteralemphasis{paramiko.SSHClient}\sphinxstyleliteralemphasis{(}\sphinxstyleliteralemphasis{)}\sphinxstyleliteralemphasis{open\_sftp}\sphinxstyleliteralemphasis{(}\sphinxstyleliteralemphasis{)}) \textendash{} SFTP connection

\item {} 
\sphinxstyleliteralstrong{folder\_to\_check} (\sphinxstyleliteralemphasis{str}) \textendash{} Path where there should be looked.

\item {} 
\sphinxstyleliteralstrong{check\_for} (\sphinxstyleliteralemphasis{str}) \textendash{} Folder, which existence is wanted.

\end{itemize}

\item[{Returns}] \leavevmode
True on success

\item[{Return type}] \leavevmode
bool

\end{description}\end{quote}

\end{fulllineitems}

\index{ssh\_connection() (in module livestock.ssh)}

\begin{fulllineitems}
\phantomsection\label{\detokenize{ssh:livestock.ssh.ssh_connection}}\pysiglinewithargsret{\sphinxcode{livestock.ssh.}\sphinxbfcode{ssh\_connection}}{}{}
This function opens up a SSH connection to a remote machine (Ubuntu-machine) based on inputs from the in\_data.txt
file. Once it is logged in then function activates the anaconda environment livestock\_env, sends the commands,
awaits their completion (by looking for a out.txt file, which is only written upon completion of the commands)
and returns the wanted files back to the local machine.

\end{fulllineitems}


\sphinxstylestrong{Go Back to:}

\sphinxhref{https://ocni-dtu.github.io/}{Livestock Frontpage}

\sphinxhref{https://ocni-dtu.github.io/livestock/index.html}{Livestock PyPi}

\sphinxhref{https://ocni-dtu.github.io/livestock\_gh/index.html}{Livestock Grasshopper}


\chapter{Documentation for the Grasshopper Components:}
\label{\detokenize{index:documentation-for-the-grasshopper-components}}\label{\detokenize{index:id3}}\begin{itemize}
\item {} \begin{description}
\item[{\sphinxhref{https://ocni-dtu.github.io/livestock\_gh/index.html}{Livestock Grasshopper Documentation}}] \leavevmode\begin{itemize}
\item {} 
\sphinxhref{https://ocni-dtu.github.io/livestock\_gh/components.html}{Components}

\item {} 
\sphinxhref{https://ocni-dtu.github.io/livestock\_gh/component\_classes.html}{Component Classes}

\item {} 
\sphinxhref{https://ocni-dtu.github.io/livestock\_gh/lib.html}{Component Library}

\end{itemize}

\end{description}

\end{itemize}

\sphinxstylestrong{Go Back to:}

\sphinxhref{https://ocni-dtu.github.io/}{Livestock Frontpage}


\chapter{Indices and tables}
\label{\detokenize{index:indices-and-tables}}\label{\detokenize{index:id5}}\begin{itemize}
\item {} 
\DUrole{xref,std,std-ref}{genindex}

\item {} 
\DUrole{xref,std,std-ref}{modindex}

\item {} 
\DUrole{xref,std,std-ref}{search}

\end{itemize}


\renewcommand{\indexname}{Python Module Index}
\begin{sphinxtheindex}
\def\bigletter#1{{\Large\sffamily#1}\nopagebreak\vspace{1mm}}
\bigletter{l}
\item {\sphinxstyleindexentry{livestock.air}}\sphinxstyleindexpageref{air:\detokenize{module-livestock.air}}
\item {\sphinxstyleindexentry{livestock.geometry}}\sphinxstyleindexpageref{geometry:\detokenize{module-livestock.geometry}}
\item {\sphinxstyleindexentry{livestock.hydrology}}\sphinxstyleindexpageref{hydrology:\detokenize{module-livestock.hydrology}}
\item {\sphinxstyleindexentry{livestock.misc}}\sphinxstyleindexpageref{misc:\detokenize{module-livestock.misc}}
\item {\sphinxstyleindexentry{livestock.ssh}}\sphinxstyleindexpageref{ssh:\detokenize{module-livestock.ssh}}
\end{sphinxtheindex}

\renewcommand{\indexname}{Index}
\printindex
\end{document}